\chapter{Background and Literature Review}
% Maybe split into two chapters?

\label{chap:refs}

\section{Machine learning fundamentals}
% In general, machine learning is the ability of an AI system to acquire its own knowledge by extracting patterns from raw data. In the last decade, one particular class of machine learning algorithms took over --- deep learning.

%Before we get to hyperparameters and optimization, we will revisit and define key concepts from machine learning so that it is clear what machine learning models we use and why.


% Briefly revisit deep neural networks and ML
Our goal is to optimize hyperparameters. But before we introduce the main topic, we need to specify the context. We are interested in supervised machine learning methods. Let $\mathcal{D}$ be a labeled dataset, consisting of examples generated independently from a data-generating distribution. Each example $(x^{(i)}, y^{i})$ consist of a feature vector $x^{(i)}\in \mathcal{X}$ and its label $y^{(i)}\in \mathcal{Y}$. Depending on what we want to predict, we distinguish classification and regression. In classification, the labels are finite-valued, while in regression the labels are real numbers. The goal of supervised learning is to train a model using the examples from the dataset $\mathcal{D}$ so that it generalizes well to data from the data-generating distribution.  % Generalization error

We train machine learning algorithms by minimizing the loss function on the training data. Two loss functions are most commonly used. For regression, the mean square error of $N$ examples is computed as $$ MSE=\frac{1}{N} \sum_{i=1}^{N} (f(x^{(i)});\theta )-y^{(i)})^2$$


In the case of classification, we use the cross-entropy loss that measures the difference between two probability distributions. Let $C$ be the number of classification classes and let $y^{(i)}$ be a distribution over all classes, which will often be just a one-hot encoding of the target class. Let the predictions of the model also form a distribution over the classes. Then the loss is calculated as follows when using mean reduction across examples $$CE=-\frac{1}{N} \sum_{i=1}^{N} \sum_{c=1}^{C} y^{(i)}_c log(f(x^{(i)});\theta )_c) $$

For neural networks, the loss function is minimized using the gradient descent algorithm. Given a loss function $L(\theta)$ that we want to minimize with respect to the model parameters, or weights, the gradient descent optimizes the function by repeatedly calculating the gradient $\nabla_\theta L(\theta)$ and performing an update to the weights in the opposite direction: $$ \theta \leftarrow \theta - \alpha \nabla_\theta L(\theta)$$

Since the goal is for the model to generalize well, we need to estimate the generalization error during training. That is why we estimate it on data not used during training. We also want to verify the performance of the final model after we choose one. Not to introduce a bias to the estimate, this should also be done on unseen data. That is why it is a best practice to split the dataset into $\mathcal{D}_{train}, \mathcal{D}_{val}, \mathcal{D}_{test}$. As we will see, using a hyperparameter optimization tool introduces yet another optimization layer, meaning we should do a similar split even for the outer optimization of hyperparameters.

% We focus on hyperparameter optimization of neural networks since neural networks are widely used today and have many hyperparameters.
A neural network describes a computation. The basic unit is called a node and the nodes are arranged in an acyclic graph. Training of the network can be split into two parts --- the forward pass and the backward pass. In the forward pass, the network produces an output from the input examples by iteratively going through the graph in a topological order. Then the loss is calculated. The gradients are computed using the backpropagation algorithm, which iterates through the nodes in reverse order and calculates partial derivatives of the loss with respect to the individual weights.



\section{Hyperparameter optimization}
% Define hyperparameters, their impact on model performance, why optimization is important

% TODO: Definition is inspired by Towards an Empirical Foundation for Assessing... Eggensperger
Let $A$ be a machine learning algorithm with hyperparameters $\lambda_1, \dots , \lambda_n$ with domains $\Lambda_1,\dots , \Lambda_n$. Let $ \mathbf{\Lambda } = \Lambda_1 \times \cdot \times \Lambda_n$ denote its hyperparameter space. Hyperparameters can be continuous, integer-valued, or categorical. Also, we can have conditional hyperparameters. We say that hyperparameter $\lambda_i$ is \emph{conditional} on another hyperparameter $\lambda_j$, if $\lambda_i$ is only active if hyperparameter $\lambda_j$ takes values from a given set $V_i(j) \subset \Lambda_j$

For each hyperparameter configuration or setting $\lambda \in \mathbf{\Lambda}$, we denote $A_\lambda$ the learning algorithm A using this hyperparameter setting. Let $\mathcal{L}(A_\Lambda, \mathcal{D}_{train}, \mathcal{D}_{valid})$ denote the validation loss of algorithm $A_\lambda$ on data $\mathcal{D}_{valid}$ when trained on $\mathcal{D}_{train}$.

\begin{defn}[Hyperparameter optimization problem]\label{defn:x}
The hyperparameter optimization problem is to find hyperparameters $\lambda^*$ that minimize the loss function  \[\lambda^*=\argmin _\lambda f(\lambda)=\argmin_\lambda \mathcal{L}(A_\lambda, \mathcal{D}_{train}, \mathcal{D}_{valid}).\]
\end{defn}

Several properties make the problem hard to solve.
\begin{itemize}
    \item It is hard to obtain derivatives of the loss function with respect to hyperparameters and we will not use them to find the optimal solution. Such an optimization problem is called a black-box optimization in literature.
    \item Each function evaluation is expensive. Fully training a single deep neural network can take days.
    \item Each function evaluation may require a variable amount of time. For example, training larger models (e.g. more artificial neurons) takes more time to train. Therefore, the hyperparameter optimization algorithm should take training time into account.
    \item Observations are noisy. Repeated training may result in models that vary in performance since it is common to use random initialization of weights. The training process itself may not be deterministic as well. For example, if we use mini-batch shuffling.
\end{itemize}

On the other hand, we can leverage parallel computation to run multiple trials at the same time. One additional benefit of solving the optimization problem limited to deep neural networks is that we have access to intermediate results.

% Note on efficiency

In this thesis, we assume that the general architecture of the neural network is already given and hyperparameters can change only smaller aspects, such as the number of neurons in a layer, or kernel size in a convolutional layer. For literature dealing with the more general problem, please refer to Neural Architecture Search (NAS).

For anyone interested in the process of hyperparameter tuning and how it might be done in practice, we recommend the Deep learning tuning playbook~\cite{tuningplaybookgithub}. The authors give valuable insights into practical aspects of hyperparameter tuning that they have collected over more than ten years of working in deep learning. These insights are rarely documented. As the authors state in the text, they could not find any comprehensive attempt to explain how to get good results with deep learning. More importantly for this thesis, it gives us insight into how experts might do hyperparameter tuning. The text reveals that even today, advanced hyperparameter tuning tools are not the ultimate solution to the problem. Instead, they recommend how to use them smartly. They propose that there is still a human expert guiding the search, at least in the first, exploratory, phase.



\section{Classic Hyperparameter Search Techniques}
% Manual search, grid search, random search
Before we get to algorithmic approaches, let us consider manual hyperparameter tuning. We cannot be surprised that people still tune hyperparameters manually. There is no technical overhead or barrier. Also, in the process of hyperparameter tuning, we gain insight into the problem, which might allow us to improve our solution in ways that are not achievable just by hyperparameter tuning. Nevertheless, there are clear limits to manual tuning so let us dive into the automated approaches. The traditional algorithms for hyperparameter optimization are grid search and random search. These algorithms are simple and still widely used.

\paragraph{Grid Search} Grid search performs an exhaustive search through a manually specified subset of the hyperparameter space. Grid search is best used when the number of hyperparameters is small, or the function evaluation is not that expensive. Its biggest drawback is that the number of configurations to evaluate grows exponentially with the number of hyperparameters. Therefore, it is best to determine which hyperparameters are the most important and limit the search only to this subset. If we did not do this, we would waste a lot of computational power on hyperparameter combinations, where only the unimportant hyperparameter changes, but the important ones stay the same. But how do we determine which hyperparameters are important and we need to tune them together? On the other hand, grid search is easily parallelizable. That is an enormous advantage since in real-world scenarios, it is not uncommon to have access to a computing cluster.

\paragraph{Random Search} Random search is often used in the HPO literature as the baseline method for more advanced algorithms. In real-world optimization problems, random search often works better than the grid search. Bergstra et al.~\cite{bergstra2012random} compared random search to grid search and found that randomly chosen trials are more efficient for hyperparameter optimization than trials on a grid. It is possible to encounter a random search with 2X-budget as a baseline in some research papers. It is just a random search with two times the budget of other methods in comparison. As Li et al.~\cite{li2018hyperband} show, 2X-budget random search provides a strong baseline.

\paragraph{Quasi-random search} If our budget is low then quasi-random search might be the better option. It works by generating a low-discrepancy sequence. Intuitively, a low-discrepancy sequence covers the whole domain evenly. Therefore, the search space is better covered even with a small number of samples. In the Deep learning tuning playbook, the authors recommend using quasi-random search over grid search and random search for the initial exploration of the hyperparameter space.



\section{Bayesian optimization}
% Going from model-less to model-based approaches
So far, we have seen model-less approaches, where each trial is independent. This approach offers some advantages, like parallelization and simplicity of implementation, but it is quite inefficient. Information obtained from previous trials is not used in any way to guide the search. Bayesian optimization methods build and use an internal model of the learning algorithm's generalization performance. Bayesian optimization is widely covered in literature, we will use the work of Brochu et al.~\cite{brochu2010tutorial} and Frazier~\cite{frazier2018tutorial} for the definitions and description of the method.

% What is BO and history
% Before mockus was Kushner 1964 and Zilinkas 1975 - 2nd page of Frazier 2018
In general, Bayesian optimization is a class of optimization methods focused on optimizing a real-valued objective function \[ \max_{x\in A} f(x). \] The method got its name from the Bayes' theorem, which is applied by stating that the posterior probability of a model M given observations E is proportional to the likelihood of E given M multiplied by the prior probability of M \[ P(M|E) \propto P(E|M)P(M).\] The foundations were laid for Bayesian Optimization by Jonas Mockus~\cite{mockus1974bayesian} in the 1970s and 1980s, but it was not until the early 2000s that Bayesian Optimization got applied to machine learning by Carl Edward Rasmussen~\cite{rasmussen2003gaussian}. Since then, it has become arguably the most prominent advanced technique for hyperparameter optimization, studied by countless groups and being implemented in many popular hyperparameter optimization frameworks.

% BO algorithm introduction
The basic loop of a Bayesian optimization algorithm is simple as shown in Algorithm \ref{alg:bo}. First, the internal model is used to suggest the next hyperparameter configuration to try. This query is much cheaper than evaluating the objective function. We can think of the suggestion as the most promising configuration given the previous observations. More precisely, the algorithm maximizes an \textit{acquisition function} $a$ defined from the \textit{surrogate model}. Then the objective function is evaluated at this configuration and the observed result is added to a database. The surrogate is updated using the database with the new observation and the process is repeated until some stopping condition is triggered, e.g.\ the algorithm runs out of budget.


\begin{algorithm}
    \caption{Bayesian Optimization}
    \begin{algorithmic}[1]
    % \State {\textbf{Initialize}:} Gaussian Process prior $GP$, acquisition function $a$, data $\mathcal{D} = \{(x_i, y_i)\}_{i=1}^n$
    \For{$t = 1, 2, ...$}
        \State Find the next point $x_t$ to evaluate: $x_t = \argmax_{x} a(x|\mathcal{D}_{1:t-1})$.
        \State Sample the objective function:  $y_t = f(x_t)+ \epsilon_t$.

        \State Augment the data: $\mathcal{D}_{1:t}= \{\mathcal{D}_{1:t-1},(x_t, y_t)\}$.
        \State Update the surrogate model given the data $\mathcal{D}_{1:t}$.
    \EndFor
    %\State \textbf{Return} the best point found (e.g., with the lowest $y$)
    \end{algorithmic}
    %\caption{}
    \label{alg:bo}
\end{algorithm}

% Practical considerations - objective function
Before we dive deeper into Bayesian Optimization, let us deal with some practical considerations and properties of the objective function and the feasible set. Since we have defined the hyperparameter optimization problem as minimization of the loss function, we can assume that $f$ is defined as $f(\lambda)=-\mathcal{L}(\lambda)$. Maximizing this function is equivalent to minimizing the original function. We also assume that the objective function is continuous. When we evaluate $f$, we observe only $f(x)$ and no first- or second-order derivatives, which would allow us to use a wider array of optimization methods. We do not know nor assume that $f$ has any special structure like concavity or linearity. To summarize, we can say that Bayesian Optimization is designed for black-box derivative-free global optimization.

% Objective assumptions - Lipschitz-continuous.
% We assume that the objective is Lipschitz-continuous. That is, there exists some constant $C$ such that for all $x_1,x_2\in A: ||f(x_1)-f(x_2)|| \leq C||x_1-x_2||$. The constant $C$ may be unknown.

% Practical considerations - feasible set
We assume further that all inputs $x$ are real-valued, which is not the case for all hyperparameters, and we will address this issue later. Finally, we assume the feasible set $A$ to be a hyper-rectangle $\{ x \in \mathbb{R}| a_i \leq x_i \leq b_i \}$, which makes it easy to assess membership.

% Where was Bayesian optimization used?

% General BO Summary
The two main components of a Bayesian Optimization algorithm are a Bayesian statistical model and an acquisition function. The model approximates the objective function including the uncertainty over its predictions and the acquisition function directly influences the sampling of configurations to evaluate. We will go through both of these components in greater detail now.


\subsection{Gaussian Process Regression}
The default model in the Bayesian Optimization literature is the GP regression. Gaussian Processes

"For continuous functions, Bayesian optimization typically works by assuming the unknown function was sampled from a Gaussian process and maintains a posterior distribution for this function as observations are made or, in our case, as the results of running learning algorithm experiments with different hyperparameters are observed (Practical Bayesian optimization 2012)"
Good overview of Gaussian processes: \cite{brochu2010tutorial}.

Practical BO of ML algorithms. They show how different kernels affect performance and describe algorithms that take into account the variable cost (duration) of learning algorithm experiments~\cite{snoek2012practical}.

Best suited for optimization over continuous domains of less than 20 dimensions and tolerates stochastic noise in function evaluations~\cite{frazier2018tutorial} Gaussian process regression is a Bayesian machine learning technique.


\subsection{Parzen-Tree Estimator}

Algorithms for hyperparameter optimization~\cite{bergstra2011algorithms} introduces TPE with Expected improvement.

\subsection{Random Forest}

Used in SMAC~\cite{hutter2010sequential}\cite{hutter2011sequential}.

\subsection{Acquisition functions}
Expected improvement
Entropy search
Knowledge gradient

\section{Multifidelity techniques}
Even though Bayesian optimization techniques are very sample-efficient, our budget might not allow for enough full training runs to find a good hyperparameter configuration. One option for dealing with such strict budget constraints is the multifidelity approach. A new hyperparameter $\lambda_{fid}$ is introduced that allows us to trade off runtime for the reliability of the information we gain from evaluation. Low fidelity values mean that the evaluations are cheap but unreliable. Fidelity close to the upper limit implies that the returned values are close to true objective values achievable with full budget. What we are most interested in is finding $\lambda_{fid}$ so that we use as few resources as possible and the rank correlation is mostly preserved between the models trained with low fidelity and the fully trained models. In other words, we want the low-fidelity models to reliably predict the ranking of the fully-trained models. In general, multifidelity HPO algorithms will start by evaluating cheap HPCs with low $\lambda_{fid}$ values and then using the remaining budget to exploit the gathered data and train the most promising models with high $\lambda_{fid}$.


\subsection{Early stopping}
% Freeze-thaw paper - extrapolating of learning curves with new kernel (infinite mixture of exponentially decaying basis functions - TODO), Bayesian optimization,
% Authors state there is possible extension to more flexible priors such as spatio-temporal GP with separable covariance
The simplest way to reduce the runtime of an algorithm is to stop the computation before it finishes. Swersky et al.~\cite{swersky2014freeze} noticed that human experts have the ability to assess whether the model will eventually be useful early in the training and developed a method to leverage early stopping. They refer to this method as freeze-thaw Bayesian optimization, because it allows for pausing or aborting the training procedure when the model does not seem promising and resuming the training later if needed. This is combined with the Bayesian optimization framework for hyperparameter search and the authors propose a technique for estimating when to pause and resume training. The algorithm uses an information-theoretic criterion to determine which models to thaw.

Freeze-thaw method relies on the assumption that for many models the training loss roughly follows an exponential decay. Swersky et al.~\cite{swersky2014freeze} developed a new kernel to serve as a prior characterizing the learning curves. This kernel is then used to forecast the final training loss and to provide these estimates to the Bayesian optimization. The kernel was successfully applied to matrix factorization and other problems, but it did not describe the learning curves of deep neural networks well. The same idea was explored by Domhan et al.~\cite{domhan2015speeding}, but with a focus on deep neural networks. They developed a technique for extrapolation of learning curves based on a probabilistic model and used the model to terminate a training run when its performance is most likely going to be worse than the performance of the best model encountered so far. They modeled learning curves using eleven different model families and concluded that even though all of these models capture certain aspects of learning curves, no single model can describe all learning curves by itself. Therefore, they combined the models in a probabilistic framework. Their approach is agnostic to the hyperparameter optimizer and sped up the hyperparameter optimization approximately by a factor of two.

\subsubsection{Hyperband}
The Hyperband algorithm, developed by Li et al.~\cite{li2018hyperband}, offers a completely different approach to dynamic resource allocation. It is a general approach with only a few assumptions. Authors originally developed it as an extension to speed up random search through early stopping.

The Hyperband internally uses the Successive Halving algorithm. Successive Halving works by uniformly allocating a budget to a set of hyperparameter configurations, then it throws out the worst half, and this process repeats until one configuration remains. The algorithm allocates exponentially more resources to more promising configurations, which usually speeds up the search considerably. In theory, the algorithm might never converge. For example, if the best configurations perform poorly in the beginning, then the algorithm discards them before they have the opportunity to converge. A practical downside of the algorithm is that we have to choose the number of configurations $n$ in the first round manually. Therefore, we have to choose between large $n$, which means training a lot of hyperparameter configurations with a smaller training time and small $n$, resulting in the exploration of fewer hyperparameter configurations that are trained for longer on average.

The optimal choice of $n$ depends mainly on how hard is it to distinguish similarly performing hyperparameter configurations from each other. We know that the intermediate losses are noisy, so we have to account for the uncertainty, which the authors of the paper call the envelope, as well. Ideally, we would wait until the envelopes do not overlap. We can observe that more resources are needed if the envelopes are wider, or if the terminal losses are closer together. The choice of $n$ also places an upper bound on the execution time of a single configuration. The more configurations we want to evaluate, the less budget is allocated to the best configurations. Therefore, there might not be enough time for them to converge, and we might select a worse hyperparameter configuration as a result.
% TODO: Is the description of Hyperband correct - HB brackets - how many Successive Halving is run inside of one HB bracket?

The Hyperband addresses the problem of selecting the optimal value of $n$ by considering several possible values of $n$ for a fixed budget $B$. Specifically, a successive halving algorithm is repeatedly called with different values of $n$. A larger value of $n$ corresponds to more aggressive early-stopping since the same amount of computational resources is distributed between more hyperparameter configurations. Furthermore, by resetting the search, Hyperband is hedging against bad instantiations of the randomly sampled configurations and their initialization.

% BOHB - their desiderata are nice, might state them in the introduction
The drawback of Hyperband is that it does not scale well into larger budgets and random search starts to close the gap. Falkner et al.~\cite{falkner2018bohb} noticed its deficiency and proposed a new algorithm BOHB to fix this. They combine Hyperband with Bayesian optimization in a way that their weaknesses are compensated by the other algorithm. Bayesian optimization needs some initial trials to gather enough data for the internal models, so it performs like a random search for a while. This is where the strong low-budget performance of Hyperband is used. On the other hand, well-fitted Bayesian optimization models provide better suggestions later in the tuning process.

% DEHB ~\cite{awad2021dehb}
Another algorithm extending the Hyperband is the DEHB developed by Awad et al.~\cite{awad2021dehb}, which uses an evolutionary optimization method instead of Bayesian optimization. More specifically, the DE stands for Differential Evolution. The evolutionary approach provides some benefits over Bayesian optimization, such as better handling of discrete dimensions, better scaling into high dimensions, and conceptual simplicity enabling easy implementation.

% TODO: Explain DEHB

The authors provided a lot of experiments in their paper, comparing DEHB to BOHB, random search, and other optimizers such as SMAC or Bayesian optimization with TPE surrogate. The benchmarks include NAS-Bench-101, NAS-HPO-Bench, or Reinforcement Learning Cartpole environment. The DEHB is much more efficient on some benchmarks while performing similarly to the BOHB, the next-best HPO optimizer in the experiments. The DEHB has very strong performance early with a low budget, and the performance does not fall off even for large budgets, which it often does for the BOHB.


\subsection{Subsampling}
The second possibility is to start the exploration with a fraction of the training data. The intuition behind this approach is that even a small subset of the training data will contain most of the information about the structure of the dataset for the model to learn. That should be enough to approximate the performance on the full dataset while training faster.

% FABOLAS
Klein et al.~\cite{klein2017fast} developed a Bayesian optimization algorithm FABOLAS. The main idea of the algorithm is to introduce subset size as an additional parameter for the Gaussian process to optimize using the acquisition function information gain per unit cost. The Gaussian process then learns to approximate the correlations between different subset size values, which allows it to efficiently use smaller subsets to accelerate the hyperparameter search. The authors performed experiments on the CIFAR10 and SVHN datasets with convolutional neural networks. FABOLAS found a good hyperparameter configuration more than 10 times faster than \xxx{MTBO} and the difference was even larger in comparison to the Hyperband. It is worth noting that after a good-performing model was found by all algorithms, the differences in test error were only minor.

% Accelerating hyperparameter optimization of a deep neural network via progressive multifidelity evaluation~\cite{zhu2020accelerating}.
A similar approach to BOHB was implemented by G. Zhu and R. Zhu~\cite{zhu2020accelerating}. They combine successive halving with progressively increasing dataset size and the number of training epochs. This way, the algorithm can explore even more configurations early on. The algorithm uses a Bayesian optimization with a surrogate model to suggest configurations for the successive halving, but the authors do not mention which surrogate model they used. To support the idea of using only a subset of the training data, the authors provide an experiment on the MNIST dataset, where they compared a LeNet trained on a full dataset versus trained only on 10\% of the dataset. The results showed a difference of a few percentage points. The authors noted, that the main difference in chosen hyperparameters was in regularization hyperparameters. That is expected since training on a smaller dataset should require stronger regularization. Finally, they compared their algorithm to BOHB on CIFAR10 and CIFAR100 datasets. In both cases the new algorithm outperformed BOHB, especially will fewer resources used.


% Auto-pytorch: multifidelity metalearning for efficient and robust autodl~\cite{zimmer2021auto}. - Uses multi-fidelity, but it's more focused on NAS

A method called DyHPO that is based on a novel Gaussian Process kernel was developed by Wistuba et al.~\cite{wistuba2022supervising}. They reviewed existing multifidielity approaches, including the Hyperband, BOHB, and DEHB. They have stated a conjecture that these gray-box methods suffer from a major issue --- low-budget performances are not always a good indicator for the full budget performances. The authors argue as an example that a properly regularized network converges slower in the first few epochs, but typically outperforms a non-regularized network after full convergence. This problem is addressed by a GP kernel capable of capturing the similarity of two hyperparameter configurations even if the configurations are evaluated on different budgets.

Another area where DyHPO should gain efficiency compared to approaches like Hyperband is that instead of pre-allocating the budget, DyHPO dynamically adapts the allocation of budgets after every HPO step. Therefore, DyHPO invests only a small budget on unpromising configurations. In the provided experiments, DyHPO showed statistically significant efficiency gains compared to other hyperparameter optimization methods such as DEHB, BOHB, and Hyperband.



Some researchers explored the idea of a two-step hyperparameter optimization method --- first optimize hyperparameters on a small subset of data and then optimize the best-performing models on the full dataset. This approach was recently studied by Yu et al.~\cite{yu2024two} on a large dataset for aerosol activation emulator, containing almost 20 million examples. Their experiment is interesting because they use random search as an optimization algorithm and focus just on the dataset sizes, trying subsets as small as 0.00025 of the whole dataset (5000 examples). They have found that it does make sense to optimize hyperparameters on a small dataset first and that a lot of good models from the low-fidelity round perform well even on the full dataset. They were able to speed up the search 135 times while using just the simple and parallel random search, albeit on a single and very specific task.



\section{Benchmarks}
In this section, we review the literature comparing and benchmarking the approaches, which should further clarify which algorithm and when to use.


Li et al.~\cite{li2018hyperband} compare the Hyperband algorithm with three Bayesian optimization algorithms (SMAC, TPE, Spearmint) on CIFAR-10, rotated MNIST and SVHN datasets. They also include random search and 2x-random search as a baseline. The Hyperband consistently outperformed other algorithms at the beginning of the search. As the search progressed to spending the whole budget, the differences were only small between the methods.

FABOLAS~\cite{klein2017fast} - FABOLAS>MTBO>Hyperband.